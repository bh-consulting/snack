\section{Radius}
\subsection{Généralités}

Radius est le protocole d'authentification le plus utilisé pour le contrôle des accès réseaux.

Dans le cadre de ce document, il sera utilisé de deux façons~:

\begin{enumerate}
\item Pour l'authentification des administrateurs sur les équipements actifs.
\item Pour le contrôle d'accès au réseau lorsqu'un utilisateur cherche à s'y relier physiquement.
\end{enumerate}

Couplé avec une base données, les droits associés aux différentes méthodes d'authentification pourront être facilement manipulables par l'intermédiaire d'une interface web.

\subsection{Choix du serveur Radius}

En se restreignant aux implémentations libres, il existe près d'une dizaine\footnote{\url{http://freeradius.org/related/opensource.html}} de solutions disponible.

Les trois dominants du marché sont~:
\begin{itemize}
\item FreeRadius
\item GnuRadius
\item OpenRadius
\end{itemize}

FreeRadius semble la solution la plus largement déployé et bénéficie d'une communauté bien supérieure aux autres solutions. La documentation est en adéquation. Enfin, une étude rapide de la date de sortie des dernières versions laisse à penser qu'il est le plus maintenu~:

\begin{description}
\item[FreeRadius :] 10/09/2012
\item[GnuRadius :] 17/12/2008
\item[OpenRadius :] 24/03/2007
\end{description}

En comptant que FreeRadius possède un paquet Debian (et a fortiori Ubuntu) pour FreeRadius, et que ça n'est pas le cas pour les deux autres solutions, la documentation\footnote{Principalement basée sur~: \url{http://www.pervasive-network.org/SPIP/Installation-de-freeradius-2-4}} qui suit sera dédiée à FreeRadius qui semble le meilleur choix en terme d'implémentation.

\subsection{Installation de FreeRadius}
\subsubsection{Paquets Debian/Ubuntu}

Installer les paquets correspondants~:
\begin{verbatim}
# apt-get install freeradius freeradius-utils
\end{verbatim}

Ajouter les machines clientes au fichier \emph{/etc/freeradius/clients.conf} (le secret \emph{poil} devra être connu du client aussi)~:
\begin{verbatim}
client serverBHConsulting {
        ipaddr = 192.168.1.2
        secret = poil
        nastype = other
}
\end{verbatim}

\subsubsection{Lien avec une base MySQL}

Installer MySQL et le paquet FreeRadius permettant de faire le lien entre le serveur d'authentification et la base de données~:
\begin{verbatim}
# apt-get install freeradius-mysql mysql-server mysql-client
\end{verbatim}

Créer les bases de données nécessaires pour FreeRadius à l'aide des schémas mis à disposition dans le paquet~:
\begin{verbatim}
# echo "create database radius;" | mysql -u root -p
# echo "grant all on radius.* to radius@'%' identified by 'motdepasse_sql'; flush privileges;" | mysql -u root -p
# mysql -uroot -p radius < /etc/freeradius/sql/mysql/schema.sql
# mysql -uroot -p radius < /etc/freeradius/sql/mysql/nas.sql
\end{verbatim}

Création d'un utilisateur simple \emph{user1}, avec une authentification via un mot de passe non-chiffré (\emph{motdepasse})~:
\begin{verbatim}
# echo "INSERT INTO radcheck(UserName,Attribute,op,Value) VALUES ('user1','Cleartext-Password',':=','motdepasse');" | mysql -u root -p radius
\end{verbatim}

Adapter le fichier \emph{/etc/freeradius/sql.conf}~:
\begin{verbatim}
sql {
        database = "mysql"
        driver = "rlm_sql_${database}"
        server = "localhost"
        login = "radius"
        password = "motdepasse_sql"
        radius_db = "radius"
        acct_table1 = "radacct"
        acct_table2 = "radacct"
        postauth_table = "radpostauth"
        authcheck_table = "radcheck"
        authreply_table = "radreply"
        groupcheck_table = "radgroupcheck"
        groupreply_table = "radgroupreply"
        usergroup_table = "radusergroup"
        deletestalesessions = yes
        sqltrace = no
        sqltracefile = ${logdir}/sqltrace.sql
        num_sql_socks = 5
        connect_failure_retry_delay = 60
        readclients = yes
        nas_table = "nas"
        $INCLUDE sql/${database}/dialup.conf
}
\end{verbatim}

Dans le fichier \emph{/etc/freeradius/radiusd.conf}, décommenter les 2 lignes suivantes~:
\begin{verbatim}
$INCLUDE sql.conf
$INCLUDE sql/mysql/counter.conf
\end{verbatim}

Créer un virtualhost Radius~:
\begin{verbatim}
# cd /etc/freeradius/sites-available
# cp default radius.bh-consulting.net
# ln -s /etc/freeradius/sites-available/radius.bh-consulting.net /etc/freeradius/sites-enabled/
\end{verbatim}

Modifier le virtualhost en éditant le fichier \emph{/etc/freeradius/sites-available/radius.bh-consulting.net}~:
\begin{verbatim}
authorize {
        preprocess
        chap
        suffix
        sql
        expiration
        logintime
        pap
}

authenticate {
        Auth-Type PAP {
                pap
        }

        Auth-Type CHAP {
                chap
        }

        eap
}

preacct {
        preprocess
        acct_unique
        suffix
}

accounting {
        detail
        radutmp
        sql
}

session {
        radutmp
        sql
}

post-auth {
        sql
        # sql_log
        exec

        Post-Auth-Type REJECT {
                attr_filter.access_reject
        }
}

...
\end{verbatim}

\subsection{Test d'authentification}

Lancer le serveur Radius en mode debug sur la machine serveur (avec l'adresse IP \texttt{192.168.1.10})~:
\begin{verbatim}
# freeradius -X
\end{verbatim}

Côté client (avec \emph{freeradius-utils} d'installé et l'IP \texttt{192.168.1.2}), tenter une authentification avec l'utilisateur créé dans la base de données~:
\begin{verbatim}
# radtest user1 motdepasse 192.168.1.10 0 poil
\end{verbatim}

Si l'authentification est réussie, le Radius est prêt à être utilisé dans le cadre des deux sections suivantes.

\subsection{Contrôle d'accès au réseau}
\subsubsection{Protocole 802.1x}

Le contrôle d'accès au réseau permet de déterminer si un utilisateur a le droit de relier sa machine au réseau ou non. Concrétement, le port du commutateur correspondant sera bloqué par défaut, et ne commencera à laisser passer le trafic qu'en cas d'authentification réussie. Il déterminera le VLAN à utiliser pour le port avant de l'ouvrir, en fonction de l'utilisateur reconnu. Ce mode de fonctionnement est permis par le protocole 802.1x décrit dans la section \ref{dot1x} page \pageref{dot1x}, qui décrit les interactions avec le serveur Radius.

\subsubsection{Configurer le commutateur}

Enregistrer le serveur Radius (avec le secret comme clé)~:
\begin{verbatim}
Switch# conf t
Switch(config)# aaa new-model
Switch(config)# radius-server host 192.168.1.1 auth-port 1812 key poil
\end{verbatim}

Configurer un port en dot1x, en suivant la procédure décrite dans la section \ref{dot1x} page \pageref{dot1x}.

Ajouter une règle d'authentification via le Radius pour le dot1x~:
\begin{verbatim}
Switch(config)# aaa authentication dot1x poil group radius
\end{verbatim}

\subsubsection{Configurer le client}

Le supplicant permet au client de s'authentifier auprès du commutateur, pour que celui-ci puisse interroger le serveur Radius au moment de la connexion physique.

Installer un supplicant~:
\begin{verbatim}
# apt-get install wpa_supplicant
\end{verbatim}

Créer le fichier \emph{/etc/wpa\_supplicant/dot1x.conf}~:
\begin{verbatim}
ctrl_interface=/var/run/wpa_supplicant
ctrl_interface_group=0
eapol_version=1
ap_scan=0

network={
        key_mgmt=IEEE8021X
        eap=MD5
        identity="user1"
        password="motdepasse"
}
\end{verbatim}

\subsubsection{Test du 802.1x via Radius}

Après avoir lancé Radius sur le serveur, lancer le supplicant sur le client~:
\begin{verbatim}
# wpa_supplicant -c /etc/wpa_supplicant/dot1x.conf -D wired -i eth0
\end{verbatim}

Relier physiquement le client au commutateur au port configuré en dot1x, et vérifier que le port passe en autorisé.

\subsubsection{Assigner un numéro de VLAN à la volée}

TODO: Ajouts dans radreply

\subsection{Authentification sur les équipements actifs}
\subsubsection{Configuration du commutateur}

Enregistrer le serveur Radius~:
\begin{verbatim}
Switch# conf t
Switch(config)# aaa new-model
Switch(config)# radius-server host 192.168.1.1 auth-port 1812 key poil
\end{verbatim}

Ajouter une règle d'authentification via le Radius pour le login en telnet~:
\begin{verbatim}
Switch(config)# aaa authentication login poil group radius
\end{verbatim}

Activer l'authentification en telnet~:
\begin{verbatim}
Switch(config)# line vty 0 15
Switch(config-line)# login authentication poil
\end{verbatim}

\subsubsection{Test d'authentification sur le commutateur}

Accéder au commutateur en telnet depuis le client~:
\begin{verbatim}
$ telnet user1@192.168.1.20
\end{verbatim}

Saisir \emph{motdepasse} en guise de mot de passe.

TODO: Comment autoriser le mode enabled ?
TODO: Mode console
