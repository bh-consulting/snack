\section{Clients NAS et SQL}

\subsection{Créer la table SQL pour les clients NAS}

Le schéma de la table est disponible dans le fichier \textit{nas.sql} :

\begin{verbatim}
mysql -u radius -p radius < /etc/freeradius/sql/mysql/nas.sql
\end{verbatim}

\subsection{Activer la lecture des clients NAS dans la table SQL 'nas'}

Pour cela, il faut modifier le fichier \textit{sql.conf}:

\begin{verbatim}
readclients = yes;
nas_table = "nas";
\end{verbatim}

\subsection{Ajouter un client NAS}

On ajoute un enregistrement à la table SQL 'nas' :

\begin{verbatim}
INSERT INTO nas VALUES (nasname, shortname, secret) VALUES ('192.168.0.254','switch','bite');
\end{verbatim}

\textbf{Note:} Pour spécifier un netmask, utiliser la valeur 192.168.0.0/24

