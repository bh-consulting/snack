\section{Machine Virtuelle}

\subsection{Instllation de VirtualBox}

\begin{verbatim}
sudo vim /etc/apt/sources.list
\end{verbatim}

Ajouter le dépôt suivant à la fin du fichier :
\begin{verbatim} 
deb http://download.virtualbox.org/virtualbox/debian precise contrib
\end{verbatim}

Installation du paquet :

\begin{verbatim}
wget -q http://download.virtualbox.org/virtualbox/debian/oracle_vbox.asc -O- | sudo apt-key add -
sudo apt-get update
sudo apt-get install linux-headers-$(uname -r) build-essential virtualbox dkms
\end{verbatim}

Installation du pack d'extension :

\begin{verbatim}
cd /tmp
wget http://download.virtualbox.org/virtualbox/4.1.12/Oracle_VM_VirtualBox_Extension_Pack-4.1.12-77245.vbox-extpack
sudo VBoxManage extpack install Oracle_VM_VirtualBox_Extension_Pack-4.1.12-77245.vbox-extpack
\end{verbatim}

\subsection{Création d'une VM Linux}

On va créer une machine virtuelle Linux Ubuntu server 12.04 avec 256Mo de mémoire vive, un disque dur virtuel de 10Gb avec le réseau en pont et une image ISO du système à installer :

\begin{verbatim}
VBoxManage createvm --name "BaseLinux" --register
VBoxManage modifyvm "BaseLinux" --memory 256 --acpi on --boot1 dvd --nic1 bridged --bridgeadapter1 eth0
VBoxManage createhd --filename BaseLinux.vdi --size 10000
VBoxManage storagectl "BaseLinux" --name "IDE Controller" --add ide
VBoxManage storageattach "BaseLinux" --storagectl "IDE Controller" --port 0 --device 0 --type hdd --medium BaseLinux.vdi
VBoxManage storageattach "BaseLinux" --storagectl "IDE Controller" --port 1 --device 0 --type dvddrive --medium /home/pi/ubuntu-12.04.1-server-amd64.iso
\end{verbatim}

Pour démarrer la VM :

\begin{verbatim}
VBoxHeadless --startvm "BaseLinux" &
\end{verbatim}

Le message suivant devrait apparaître dans la console :

\begin{verbatim}
VRDE server is listening on port 3389.
\end{verbatim}

On peut se connecter à la VM via rdestock :

\begin{verbatim}
rdesktop 193.50.40.114:3389
\end{verbatim}

\subsection{Duplication de VM}

On clone une VM déjà installée :

\begin{verbatim}
VBoxManage clonevm BaseLinux --name Linux1 --register
\end{verbatim}

\subsection{Supprimer une VM}

On doit la désincrire et supprimer les fichier :

\begin{verbatim}
VBoxManage unregistervm Linux1 --delete
\end{verbatim}

\subsection{Divers}

Retirer un disque du lecteur dvd :

\begin{verbatim}
VBoxManage modifyvm BaseLinux --dvd none
\end{verbatim}
