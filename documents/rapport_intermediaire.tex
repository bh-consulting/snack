\newcommand{\titreA}{Rapport intermédiaire}
\newcommand{\titreB}{Projet industriel}
\documentclass[12pt]{article}
\usepackage[utf8]{inputenc}
\usepackage[francais]{babel}
\usepackage[T1]{fontenc}
\usepackage{lmodern, marvosym, geometry, graphicx, multicol, lastpage, tikz, listings}
\usepackage[hyperindex=true, colorlinks=true, breaklinks=true, linkcolor=blue]{hyperref}
\usepackage{fancyhdr, verbatim, alltt, pdfpages}
\usepackage{color}

\geometry{hmargin=1.5cm, vmargin=2cm}
\addtolength{\parskip}{10pt}
\pagestyle{fancy}

\definecolor{lightgray}{gray}{0.9}
\definecolor{darkgray}{gray}{0.4}
\newcommand{\hlc}[1]{\color{purple}{\textbf{#1}}}
\lstset{language=,
    basicstyle=\sffamily\footnotesize,
    xleftmargin=20pt,
    xrightmargin=20pt,
    numbers=left,
    stepnumber=1, % le pas des numeros de ligne
    numbersep=10pt,
    breaklines=true,
    tabsize=4,
    %frame=single,
    showspaces=false,
    showstringspaces=false
    inputencoding=utf8, %put1 d'utf8 dans listing
    extendedchars=true, %put1 d'utf8 dans listing
    literate={à}{{\`a}}1 {é}{{\'e}}1 {è}{{\`e}}1 {ê}{{\^e}}1 {â}{{\^a}}1 {î}{{\^i}}1 {ê}{{\^e}}1 {É}{{\'E}}1 {À}{{\`A}}1 {«}{{\og}}1 {»}{{\fg{}}}1 {ô}{{\^o}}1 {ù}{{\`u}}1 {û}{{\^u}}1 {ç}{{\c{c}}}1 {Ç}{{\c{C}}}1 {--}{{-\,-}}1 {-}{{-}}1 {*}{{*}}1 {Switch(config)\#}{{\textbf{Switch(config)\# }}}1 {Switch(config-if)\#}{{\textbf{Switch(config-if)\# }}}1 {Switch(config-line)\#}{{\textbf{Switch(config-line)\# }}}1 {Switch>}{{\textbf{Switch> }}}1 {Switch\#}{{\textbf{Switch\# }}}1, %put1 d'utf8 dans listing
    columns=fullflexible, % suppression des espaces autour des deux-points
    escapechar=§, % permet d'insérer du §code latex§
    backgroundcolor=\color{lightgray},
    alsoletter={.,1,2,3,4,5,6,7,8,9,0},
    keywords={path_to_certs, user_login, user_password, nom_entreprise, pass_radius_sql, utilisateur1, pass_utilisateur1, secret_radius, 192.168.1.2, 192.168.1.10, 255.255.255.0, fastethernet0/1, vlan_id, pass_enable, pass_enable_secours, utilisateur_secours, pass_utilisateur_secours, pass_root_sql, snack, 192.168.0.254, commutateur1, client1, Nancy, BHConsulting, France, FR, nom_serveur_radius, dossier_certs, dossier_certs_utilisateur, eth0, path_to_backups, path_to_script, },
    keywordstyle=\hlc,
    morecomment=[l]{\#}, % commentaires shell en ligne
    commentstyle=\itshape\color{darkgray},
}

\renewcommand{\headrulewidth}{1pt}
\lhead{\textbf{SNACK (\titreB)}}
\rhead{\emph{BOUGET / GUÉPIN / PINHÈDE / VAUBOURG}}
\lfoot{TELECOM Nancy - PI}
\cfoot{\thepage{} / \pageref{LastPage}}
\rfoot{2012-2013}

\begin{document}
\thispagestyle{empty}

\begin{multicols}{2}
{\large
	\begin{flushleft}
		\noindent{}\textbf{Nicolas BOUGET}\\
		\Letter~nicolas.bouget@esial.net\\
		3A TRS1\\~

		\noindent{}\textbf{Julien GUÉPIN}\\
		\Letter~julien.guepin@esial.net\\
		3A IL\\
	\end{flushleft}

	\begin{flushright}
		\noindent{}\textbf{Marc PINHÈDE}\\
		\Letter~marc.pinhede@esial.net\\
		3A LE\\~

		\noindent{}\textbf{Julien VAUBOURG}\\
		\Letter~julien@vaubourg.com\\
		3A TRS2\\
	\end{flushright}
}
\end{multicols}

\vspace{0.5cm}

\begin{center}
	{\Huge\textbf{\titreA}}

	\vspace{1cm}

	{\huge\emph{\titreB}}

	\vspace{1cm}

	\begin{flushleft}
		{\large
		\hspace{3.2cm}
		\textbf{Société~:} B.H. Consulting\\
		\hspace{3.2cm}
		\textbf{Intervenant industriel~:} Guillaume ROCHE\\
		\hspace{3.2cm}
		\textbf{Intervenant universitaire~:} Jean-François SCHEID
		}
	\end{flushleft}

	\vspace{1cm}
	{\large Le \today}

	\vspace{1.5cm}

	\includegraphics[width=140pt]{img/BHConsulting.jpg}

	\vspace{1.2cm}

	\includegraphics[width=190pt]{img/ul.png}
	\hspace{3.5cm}
	\includegraphics[width=140pt]{img/telecom-nancy.jpg}
\end{center}
\newpage

\thispagestyle{empty}
\tableofcontents
\newpage



\section{Introduction}
\section{B.H. Consulting}
\section{Projet}
\subsection{Contexte}

La sensibilisation croissante aux notions de sécurisation des réseaux des entreprises a conduit B.H. Consulting à devoir répondre à de nouveaux types d'exigences.

Les réseaux d'une entreprise sont en effet devenus un point critique de toutes les nouvelles sociétés, qui sont régulièrement amenées à perdre des heures considérables de travail dès lors que ceux-ci ne sont plus accessibles. Au-delà de l'aspect fonctionnel, les réseaux sont une porte vers les serveurs les plus sensibles des administrations, qui y stockent les informations les plus confidentielles.

Qu'ils soient filaires ou non, ces réseaux sont accessibles dès lors qu'on est dans les locaux des entreprises~: prises murales, émissions wifi, etc. N'importe quel individu ayant la possibilité de se relier physiquement aux réseaux, il faut mettre en place un mécanisme de contrôle logique pour restreindre ses accès. Pour autant, ce mécanisme ne doit pas être lourd à administrer, ce qui impliquerait rapidement un laxisme et une confusion dans les règles établies. Définir strictement les autorisations en fonction des points d'accès physiques, pour ensuite établir une carte des autorisations de l'ensemble des bâtiments n'est pas non plus la bonne solution, puisqu'il sera difficile de garantir les contrôles d'accès à ces prises. L'utilisation du réseau dépendra également de  l'emplacement géographique de l'utilisateur, à l'encontre de toute logique vis-à-vis de la tendance actuelle au nomadisme des entreprises.

Pour répondre à toutes ces exigences, critiques pour une entreprise qui souhaite garantir l'intégrité de son réseau et la confidentialité de ses données tout en sauvegardant la liberté de mouvement de ses employés, B.H. Consulting est régulièrement contraint d'exploiter de nouvelles technologies adéquates. La première problématique de ce projet industriel sera donc de permettre à B.H. Consulting d'avoir la documentation nécessaire pour déployer l'une des solutions existantes facilement et rapidement.

Au delà du déploiement, B.H. Consulting souhaite offrir une souplesse d'administration de son système exemplaire, en permettant aussi bien à ses clients que ses propres employés de modifier les permissions des réseaux. Aucune des solutions d'interface existantes ne répondant à ce besoin actuellement, il s'agit du second objectif de ce projet industriel. L'interface devra devra donc être accessible, ergonomique et complète. Des fonctionnalités d'administration poussées comme la gestion d'un historique, d'une console virtuelle et la gestion des équipements actifs la compléteront durant ce projet.

Les objectifs de ce projet industriel émanent directement des besoins des clients, auxquels B.H. Consulting doit faire face.

\subsection{Objectifs}

Plusieurs objectifs peuvent être tirés des problématiques sus-évoquées~:

\begin{enumerate}
\item Définir la technologie qui permettra de définir des accès aux réseaux de façon dynamique, souple et sécurisée.
\item Former l'équipe du projet industriel sur cette technologie, à l'aide de documentations et d'expérimentations concrétes sur les différents aspects sous-jacents.
\item Écrire une documentation sur le sujet qui sera restreinte et utile à B.H. Consulting dans le cadre de ses besoins.
\item Concevoir une solution logicielle permettant le déploiement rapide et efficace de la technologie sur les réseaux des clients.
\end{enumerate}

Une seconde liste d'objectifs de dégage rapidement, et de façon relativement indépendante de la première~:

\begin{enumerate}
\item Concevoir une interface permettant la configuration des accès au réseau, avec des profils d'utilisateurs définis par des ensembles de permissions.
\item Ajouter des fonctionnalités supplémentaires à l'interface, comme la console ou la gestion des équipements actifs.
\item Intégrer l'interface dans la solution logicielle proposée lors de la première étape, pour le déploiement de la solution chez les clients.
\end{enumerate}

Afin de répondre aux attentes de l'entreprise de façon optimale, en ayant l'assurance d'être exhaustif et de soulever toute ambiguïté, les objectifs ont nécessité d'être contractualisés. Les documents qui en résultent ont été rédigés en partenariat avec les responsables de l'entreprise, en respectant les étapes classiques d'une gestion de projet.

\section{Gestion de projet}
\subsection{Constitution de l'équipe}

L'équipe du projet est constituée exclusivement d'élèves-ingénieurs de l'école TELECOM Nancy, en dernière année de formation. Après avoir remporté la confiance de l'entreprise pour se faire attribuer ce travail, le chef de projet a confirmé l'équipe suivante~:

\begin{description}
\item[Nicolas BOUGET] Ayant intégré la spécialisation \textit{Télécoms, Réseaux et Services} (TRS) dès la seconde moitié de la deuxième année de son cursus ingénieur, Nicolas a une sensibilité particulière pour le réseau et l'administration de services. Avec une expérience forte de conception de tests pour un simulateur d'avion, acquise durant son stage de deuxième année, Nicolas dispose d'une rigueur et d'une méthodologie reconnue, avec une sensibilisation aux problématiques du développement de gros projets. Co-responsable du parc informatique de la convention de culture japonaise Anim'Est lors de sa dernière édition, il est également capable de gérer des infrastructures imposantes en supportant un stress important et une nécessité de qualité de service exemplaire. C'est donc naturellement qu'il a été désigné comme chef de projet, dans le cadre de ce travail.\\
\item[Julien GUÉPIN] Julien a choisi la spécialisation \textit{Ingénieurie Logicielle} (IL) lors de sa seconde année à TELECOM Nancy. Véritable passionné de développement web, Julien apporte des compétences extrêmement fortes dans un projet qui nécessite une expertise sérieuse pour aboutir à une solution logicielle munie d'une interface web fonctionnelle, ergonomique et efficace. Avec de nombreuses expériences dans le domaine, il a été capable de prouver ses capacités de nombreuses fois dans le passé, autant au travers de ses stages que ses projets personnels. Actif associativement, Julien s'est notamment investi dans un projet humanitaire au Pérou, développant ainsi des capacités évidentes de travail en équipe et d'organisation personnelle pour conduire un projet complexe vers une réussite reconnue. Particulièrement intéressé par la seconde partie du projet, il est un élément clé de la réussite de ce projet.\\
\item[Marc PINHÈDE] C'est une troisième spécialité offerte par l'école que Marc propose en intégrant l'équipe, puisqu'il a choisi dès l'année précédente l'option \textit{Logiciels Embarqués} (LE). Sa passion pour le domaine, ainsi que la rigueur imposée par les environnements embarqués font de lui une ressource particulièrement minutieuse et attachée à la perfection, qui conduit à la performance et l'excellence. Marc a notamment eu l'occasion de prouver ses compétences lors de son stage de seconde année dans un centre de recherche, en y implémentant un compilateur. Avec des capacités d'intégration et de travail en équipe reconnues à chacune de ses implications dans la convention Anim'Est, Marc est un membre essentiel de l'équipe, qui saura mettre à profit ses compétences dans les domaines couverts par ce projet, qui l'intéressent tout autant.\\
\item[Julien VAUBOURG] Également issue de la spécialisation TRS, Julien est un passionné des réseaux qui dispose en plus d'une forte expérience dans le développement. Une double-compétence qu'il a su prouver à plusieurs reprises au travers de ses nombreux stages et projets personnels. Particulièrement impliqué associativement, notamment dans l'administration d'un fournisseur d'accès à Internet participatif, Julien a l'occasion d'intégrer très régulièrement des équipes, tout en étant force de proposition. Également acteur de la dernière édition de la convention Anim'Est auprès de Nicolas et Marc, il intègre naturellement l'équipe avec une capacité de vision d'ensemble du projet.
\end{description}

Il s'agit donc d'une équipe particulièrement pluridisciplinaire, qui saura composer avec ses membres pour aggréger les compétences de chacun pour obtenir le meilleur d'eux-mêmes, dans l'optique de former un tout très complet. Tous les membres ayant déjà eu l'occasion de travailler ensemble, l'équipe est dynamique et prête à travailler dès le lancement du projet.

\subsection{Note de cadrage}
\subsection{Cahier des charges}
\subsection{Réunions}

\section{Avancement}

\section{Conclusion}

\end{document}
