\section{Sauvegarde des configurations}
\subsection{Serveur}

Installer les paquets suivants en super-utilisateur~:

\begin{lstlisting}
$ sudo apt-get install snmp snmpd git tftpd-hpa
\end{lstlisting}

Puis, configurer les paquets comme décris ci-dessous.

\subsubsection{radius}

dans /etc/freeradius/sites-enabled/snack, completez la section accounting par l'appel conditionnel d'un module snack-backups~:
\begin{lstlisting}
    accounting{
	if(((NAS-Port-Type == Async)||(NAS-Port-Type == Virtual))&&((Acct-Status-Type == Start)||(Acct-Status-Type == Stop))) {
	    snack-backups
	}
	detail
	...
\end{lstlisting}

et créer le-dit module snack-backups, en créant le fichier /etc/freeradius/modules/snack-backups comme suit~:
\begin{lstlisting}
exec snack-backups {
            program = "path_to_script/backup_create.sh"
            wait = no
            input_pairs = request
            shell_escape = yes
            output = none
}
\end{lstlisting}
Le path\_to\_script correspond au chemin vers le script 'backup\_create.sh' donné en annexe. Nous utilisons '/home/snack/scritps'.


\subsubsection{snmptrapd}

Configurer le fichier /etc/snmp/snmptrapd.conf, initalement vide, de la façon suivante~:
\begin{lstlisting}
	donotlogtraps false
	logOption f /var/log/snmptraps.log
	authCommunity log,execute,net private
	traphandle default path_to_script/backup_traps.sh
\end{lstlisting}

Puis forcer le lancement de snmptrapd avec snmp en corrigant /etc/init.d/snmpd {\color{red}ET} /etc/default/snmpd. La variable TRAPDRUN dois prendre la valeur 'yes' au lieu de 'no'~:

\begin{lstlisting}
    SNMPDRUN=yes
    SNMPDOPTS='-Lsd -Lf /dev/null -p /var/run/snmpd.pid'
    TRAPDRUN=yes
    TRAPDOPTS='-Lsd -p /var/run/snmptrapd.pid'
\end{lstlisting}

Puis redémarrer le service~:

\begin{lstlisting}
$ sudo killall snmaptrapd
$ sudo snmptrapd 
\end{lstlisting}

\subsubsection{tftp}

Créer un dossier réservé aux sauvegardes de configurations. Nous utilisons par défault le path\_to\_backups '/home/snack/.

\begin{lstlisting}
    mkdir path_to_backups/backups.git
\end{lstlisting}


Editer le fichier /etc/init/tftpd-hpa.conf pour forcer l'écoute sur le bon dossier~:

\begin{lstlisting}
script

    if [ -f ${DEFAULTS} ]; then
        . ${DEFAULTS}
    fi
    
    if [ ! -d "${TFTP_DIRECTORY}" ]; then
	echo "${TFTP_DIRECTORY} missing, aborting."
	exit 1
    fi
    
    
    exec /usr/sbin/in.tftpd --listen --address 0.0.0.0:69 --secure path_to_backups/backups.git -c"\
end script
\end{lstlisting}
l'option '--user username' peut-être utilisée pour force tftp à prendre l'UID d'un utilisateur donné lors de l'écriture.\\

Modifier aussi la variabl 'TFTP\_DIRECTORY' du fichier /etc/default/tftpd-hpa, pour que les NAS puissent lire le bon dossier et récuperer d'ancienne configurations~:
\begin{lstlisting}
TFTP_USERNAME="tftp"
TFTP_DIRECTORY="path_to_backups/backups.git"
TFTP_ADDRESS="0.0.0.0:69"
TFTP_OPTIONS="--secure"
\end{lstlisting}

Puis redémarrer le service~:
\begin{lstlisting}
sudo service tftpd-hpa stop

/usr/sbin/in.tftpd --listen --address 0.0.0.0:69 --secure path_to_backups/backups.git -c
\end{lstlisting}

\subsubsection{git}

Initialiser le dépôt git dans le répertoire créé dans la section précedente~:

\begin{lstlisting}
~ $ cd path_to_backups/backups.git
backups.git $ git init
Initialized empty Git repository in path_to_backups/backups.git/.git/
\end{lstlisting}


\subsubsection{Interface}

Les scripts de mise à jours utilisent l'adresse IP du radius. Celle ci peut être réglée via l'interface ou directement en modifiant /home/snack/interface/app/Config/parameters.php:
\begin{lstlisting}
    ...
    'configurationEmail' => 'pi@bh.net',
    'errorEmail' => 'pi@bh.net',
    'ipAddress' => '192.168.1.10',
    'paginationCount' => '10',
    ...
\end{lstlisting}

\subsubsection{Restauration}
Pour la restauration de configuration, il suffit d'utiliser le script 'restore', donné en annexe.


\subsection{Commutateur}

En mode de configuration du terminal, activer l'envoi des trappes SNMP sur le commutateur cisco~:

\begin{lstlisting}
Switch(config)# snmp-server community private rw
Switch(config)# snmp-server host 192.168.1.10 version 2c private
Switch(config)# snmp-server enable traps config
Switch(config)# snmp-server enable traps snmp coldstart
\end{lstlisting}


\subsection{Correction de l'accounting}

Nos observations nous ont permis de découvrir que les sessions d'accounting étaient écrasées par radius lors d'un 'reload' du commutateur. Ceci est dû au fait que les sessions accounting sont identifiées par un id supposé unique envoyé par le commutateur. Cependant, celui-ci re-utilise les même id une fois re-démarré.\\
Si la sauvegarde des sessions est appliquée, et en particulier le redémarrage des équipements actifs monitorés, ce comportement peut être corrigé coté radius.\\

Modifier le fichier /etc/freeradius/modules/acct\_unique, en forcant l'utilisation de Tmp-String-0 dans la génération de la clé~:
\begin{lstlisting}
acct_unique {
        key = "User-Name, Acct-Session-Id, NAS-IP-Address, Client-IP-Address, NAS-Port, Tmp-String-0"
	}
\end{lstlisting}

puis le fichier /etc/freeradius/sites-enabled/snack, section preacct, ou on affecte à Tmp-String-0 la date du dernier reload~:
\begin{lstlisting}
preacct {
	update request {
		Tmp-String-0 := "%{sql:SELECT datetime FROM backups WHERE nas='%{NAS-IP-Address}' AND action='boot' ORDER BY datetime DESC LIMIT 1}"
	}
	preprocess
	...
\end{lstlisting}

enfin, dans le fichier /etc/freeradius/sql/mysql/dialup.conf, forcer l'utilisation de l'acctuniqueid au lieu de l'acctsessionid~:
\begin{lstlisting}
...
	accounting_update_query = " \
	UPDATE ${acct_table1} \
	SET \
	    framedipaddress = '%{Framed-IP-Address}', \
	    acctsessiontime     = '%{Acct-Session-Time}', \
	    acctinputoctets     = '%{%{Acct-Input-Gigawords}:-0}'  << 32 | \
				  '%{%{Acct-Input-Octets}:-0}', \
	    acctoutputoctets    = '%{%{Acct-Output-Gigawords}:-0}' << 32 | \
				  '%{%{Acct-Output-Octets}:-0}' \
	WHERE §\textbf{acctuniqueid}§    = '%{§\textbf{Acct-Unique-Session-ID}§}' \
	AND username        = '%{SQL-User-Name}' \
	AND nasipaddress    = '%{NAS-IP-Address}'"
...
	accounting_start_query_alt  = " \
	UPDATE ${acct_table1} SET \
	    acctstarttime     = '%S', \
	    acctstartdelay    = '%{%{Acct-Delay-Time}:-0}', \
	    connectinfo_start = '%{Connect-Info}' \
	WHERE §\textbf{acctuniqueid}§    = '%{§\textbf{Acct-Unique-Session-ID}§}' \
	AND username         = '%{SQL-User-Name}' \
	AND nasipaddress     = '%{NAS-IP-Address}'"
...
	accounting_stop_query = " \
	UPDATE ${acct_table2} SET \
	    acctstoptime       = '%S', \
	    acctsessiontime    = '%{Acct-Session-Time}', \
	    acctinputoctets    = '%{%{Acct-Input-Gigawords}:-0}' << 32 | \
				 '%{%{Acct-Input-Octets}:-0}', \
	    acctoutputoctets   = '%{%{Acct-Output-Gigawords}:-0}' << 32 | \
				 '%{%{Acct-Output-Octets}:-0}', \
	    acctterminatecause = '%{Acct-Terminate-Cause}', \
	    acctstopdelay      = '%{%{Acct-Delay-Time}:-0}', \
	    connectinfo_stop   = '%{Connect-Info}' \
	WHERE §\textbf{acctuniqueid}§    = '%{§\textbf{Acct-Unique-Session-ID}§}' \
	AND username          = '%{SQL-User-Name}' \
	AND nasipaddress      = '%{NAS-IP-Address}'"
...
\end{lstlisting}



