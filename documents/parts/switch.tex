\section{Commutateur Cisco (NAS)}
\subsection{Authentification des clients en 802.1x}
\subsubsection{Paramètres par défaut}

{\Large\Info} Le commutateur utilisé pour rédiger cette documentation est un Catalyst 2950 (C2950-I6Q4L2-M) avec un IOS en version 12.1(22)EA14. Il est possible que certaines commandes différent légèrement d'une version d'IOS à l'autre. Toutes les versions de IOS ne supportent pas le protocole 802.1x\footnote{Voir \url{http://tools.cisco.com/ITDIT/CFN/jsp/by-feature-technology.jsp}}.

La configuration par défaut est~:

\begin{itemize}
\item Authentication, authorization, and accounting (AAA) authentication <-> Disabled
\item RADIUS server : IP address <-> None specified
\item RADIUS server : UDP authentication port <-> 1812
\item RADIUS server : Key <-> None specified
\item Per-interface 802.1X enable state <-> Disabled (force-authorized)
\item Periodic re-authentication <-> Disabled
\item Number of seconds between re-authentication attempts <-> 3600 seconds
\item Quiet period <-> 60 seconds
\item Retransmission time <-> 30 seconds
\item Maximum retransmission number <-> 2 times
\item Multiple host support <-> Disabled
\end{itemize}

Par défaut, le protocole 802.1x est désactivé.

\subsubsection{Activer le protocole 802.1x}

Activer le protocole sur le commutateur~:

\begin{lstlisting}
Switch> en
Switch# conf t
Switch(config)# aaa new-model
Switch(config)# aaa authentication dot1x default group radius
Switch(config)# dot1x system-auth-control
\end{lstlisting}

\subsubsection{Déclarer un serveur Radius}

Ajouter la définition d'un serveur Radius (\emph{192.168.1.10})~:

\begin{lstlisting}
Switch(config)# aaa new-model
Switch(config)# radius-server host 192.168.1.10 auth-port 1812 key secret_radius
\end{lstlisting}

Pour que le commutateur puisse communiquer avec le Radius, il faut lui définir une adresse IP~:

\begin{lstlisting}
Switch(config)# interface vlan 1
Switch(config-if)# no shutdown
Switch(config-if)# ip addr 192.168.1.2 255.255.255.0
Switch(config-if)# exit
\end{lstlisting}

\textbf{{\huge\Stopsign} Attention~:} L'adresse IP du NAS doit être la même que le client (qui jouait le rôle de NAS) lors des tests précédents. Sinon, son adresse doit être ajoutée à la table des NAS du serveur Radius (cf. section~\ref{ajout-nas} page~\pageref{ajout-nas}).

\subsubsection{Configurer une interface}

Passer l'interface en mode \emph{access}~:

\begin{lstlisting}
Switch(config)# interface §\hlc{fastethernet0/1}§
Switch(config-if)# switchport mode access
Switch(config-if)# no shutdown
\end{lstlisting}

Le protocole 802.1x ne peut pas être activé sur les ports suivants~:

\begin{enumerate}
\item Trunk
\item Dynamic
\item Dynamic-access
\item Etherchannel
\item Secure
\item Switch Port Analyzer (SPAN)
\end{enumerate}

Activer le 802.1x pour cette interface en particulier~:

\begin{lstlisting}
Switch(config-if)# dot1x port-control auto
Switch(config-if)# exit
\end{lstlisting}

\subsubsection{Test de l'authentification en 802.1x}

Relier le client en ethernet au port configuré précédemment. Par défaut, le voyant du port devrait être orange sur le commutateur et le client ne devrait pas pouvoir accéder au réseau.

Réutiliser le fichier d'authentification créé pour le test d'authentification simple en MD5 de la section~\ref{auth-md5} page~\pageref{auth-md5}.

Cette fois-ci il faut tester l'authentification avec un vrai \emph{supplicant}, comme WpaSupplicant~:

\begin{lstlisting}
$ sudo wpa_supplicant -c dot1x.md5.conf -D wired -i eth0
\end{lstlisting}

Grâce au Freeradius lancé en mode en mode verbeux sur le serveur (cf. section~\ref{freeradius-x} page~\pageref{freeradius-x}), une authentification réussie devrait afficher, comme lors des autres tests~:

\begin{lstlisting}[morekeywords={0x03010004, 8}]
Sending Access-Accept of id 8 to 192.168.1.2 port 1812
        EAP-Message = 0x03010004
        Message-Authenticator = 0x00000000000000000000000000000000
        User-Name = "utilisateur1"
\end{lstlisting}

Du côté du commutateur, le voyant du port ethernet devrait passer en jaune~: le réseau est accessible pour le client, qui peut dès lors récupérer une IP en DHCP.

Pour tester l'authentification par certificat, utiliser les fichiers proposés dans la section~\ref{tests-certificats} page~\pageref{tests-certificats} directement avec la commande \emph{wpa\_supplicant} donnée ci-dessus.

Exemple d'envoi de réponse positive avec TLS~:

\begin{lstlisting}[morekeywords={0xb6d79d7dc6ac237bade516921f3c801e790ae968b0e83de5c4ad8c727df824ea, 0x41345ea21c13773a0f4f4d279b170fe0deb9cb37f886c4c19f55dd079e92d1e2, 0x03090004, 28}]
Sending Access-Accept of id 28 to 193.50.40.71 port 1812
        MS-MPPE-Recv-Key = 0xb6d79d7dc6ac237bade516921f3c801e790ae968b0e83de5c4ad8c727df824ea
        MS-MPPE-Send-Key = 0x41345ea21c13773a0f4f4d279b170fe0deb9cb37f886c4c19f55dd079e92d1e2
        EAP-Message = 0x03090004
        Message-Authenticator = 0x00000000000000000000000000000000
        User-Name = "utilisateur1"
\end{lstlisting}

\subsubsection{Debug}

Pour obtenir des informations concernant le fonctionnement actuel du 802.1x côté commutateur~:

\begin{lstlisting}
Switch> enable
Switch# debug dot1x all
Switch# conf t
Switch(config)# show dot1x
Switch(config)# show dot1x statistics
\end{lstlisting}

\subsection{Authentification des administrateurs du commutateur}
\subsubsection{Identifiants Radius}

Commencer par ajouter le mode d'authentification général~:

\begin{lstlisting}
Switch> enable
Switch# configure terminal
Switch(config)# aaa authentication login default group radius
\end{lstlisting}

Pour l'activer sur le port console (penser à faire un \emph{write mem} avant, pour pouvoir relancer le commutateur sans perdre les modifications précédentes si la console devenait inaccessible)~:

\begin{lstlisting}[morekeywords=0]
Switch(config)# line console 0
Switch(config-line)# login authentication default
Switch(config-line)# exit
\end{lstlisting}

Tester une authentification sur le port console, avec l'utilisateur \emph{utilisateur1} et le mot de passe \emph{pass\_utilisateur1} (créé dans la section~\ref{ajout-utilisateur-md5} page~\pageref{ajout-utilisateur-md5}). % Pour activer le mode \emph{enable}, voir la section suivante.

Pour l'activer sur les connexions Telnet/SSH~:

\begin{lstlisting}[morekeywords=0]
Switch(config)# line vty 0
Switch(config-line)# login authentication default
Switch(config-line)# exit
\end{lstlisting}

Tester une authentification en Telnet ou SSH, avec les mêmes identifiants que précédemment.

Les authentifications devraient fonctionner, mais pas le passage en super-utilisable avec la commande \emph{enable}. L'accès à ce privilège requiert des actions supplémentaires sur le serveur Radius.

\subsubsection{Activation du mode \emph{enable}}

Activer l'authentification pour le passage en super-utilisateur sur le commutateur~:

\begin{lstlisting}[morekeywords=0]
Switch(config)# aaa authentication enable default group radius
\end{lstlisting}

Sur le serveur Radius, ajouter cette ligne dans la section \emph{sql} du fichier \emph{/etc/freeradius/sql.conf}~:

\begin{lstlisting}
sql {
	§\emph{[...]}§

	safe-characters = "@0123456789abcdefghijklmnopqrstuvwxyzABCDEFGHIJKLMNOPQRSTUVWXYZ0123456789.-_: /$"
}
\end{lstlisting}

Puis ajouter cet utilisateur particulier dans la base de données, qui définit le mot de passe pour passer en super-utilisateur~:

\begin{lstlisting}
$ sudo echo "INSERT INTO radcheck(username, attribute, op, value) VALUES ('\$enab15\$', 'Crypt-Password', ':=', encrypt('pass_enable'))" | mysql -uradius -p§§pass_radius_sql radius
\end{lstlisting}

Tester un passage en mode \emph{enable} dans une connexion en mode console, Telnet ou SSH.

\subsubsection{Identifiants de secours}

Si jamais le serveur Radius n'est pas joignable, personne ne pourra se connecter au commutateur pour le configurer. Lorsqu'un réseau tombe en panne, il est probable qu'il faille configurer le commutateur alors que le serveur Radius n'est plus accessible. C'est pourquoi il faut installer des identifiants de secours.

Suivre la procédure suivante~:

\begin{lstlisting}
Switch>en
Switch#conf t
Switch(config)#aaa authentication login default group radius local
Switch(config)#aaa authentication enable default group radius enable
Switch(config)#username utilisateur_secours password pass_utilisateur_secours
Switch(config)#enable password pass_enable_secours
\end{lstlisting}

Tenter une authentification (console, Telnet, SSH ou directement en série) avec l'utilisateur \emph{utilisateur\_secours} et le mot de passe \emph{pass\_utilisateur\_secours}. Puisque le serveur Radius est accessible, elle ne devrait pas fonctionner. Couper le Freeradius (Ctrl-C) et retenter~: l'authentification fonctionne (après un certain temps correspondant à l'attente nécessaire pour le commutateur conclut que le Radius n'est pas accessible) 

Le mot de passe \emph{pass\_enable\_secours} pourra être utilisé pour passer en mode privilégié lorsque le Radius n'est pas disponible.

\textbf{Félicitations, penser à faire un \emph{write mem} final \Coffeecup.}
