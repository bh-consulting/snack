\section{Maintenance}
\subsection{Ajouter un client NAS}

Pour ajouter un client NAS (en général un commutateur), qui pourra dès lors interroger le Radius~:

\begin{lstlisting}
$ echo "INSERT INTO nas VALUES (nasname, shortname, secret) VALUES ('192.168.0.254', 'commutateur1', 'secret_radius')" | mysql -uroot -p§§pass_radius_sql radius
\end{lstlisting}

Pour spécifier un masque de sous-réseau, utiliser la notation CIDR.

Pour supprimer un NAS~:

\begin{lstlisting}
$ echo "DELETE FROM nas WHERE shortname='commutateur1' | mysql -uroot -p§§pass_radius_sql radius
\end{lstlisting}

Pour ajouter un utilisateur~:

\begin{lstlisting}
$ echo "INSERT INTO radcheck(username, attribute, op, value) VALUES ('utilisateur1', 'Cleartext-Password', ':=', 'pass_utilisateur1')" | mysql -uroot -p§§pass_radius_sql radius
\end{lstlisting}

Penser à lui générer un certificat.

Pour supprimer un utilisateur~:

\begin{lstlisting}
$ echo "DELETE FROM radcheck WHERE username='utilisateur1' AND value='pass_utilisateur1'" | mysql -uroot -p§§pass_radius_sql radius
\end{lstlisting}
