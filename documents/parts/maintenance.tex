\section{Maintenance}
\subsection{Ajouter un client NAS}

Pour ajouter un client NAS (en général un commutateur), qui pourra dès lors interroger le Radius~:

\begin{lstlisting}
$ echo "INSERT INTO nas VALUES (nasname, shortname, secret) VALUES ('192.168.0.254','switch','§\emph{mysecret}§')" | mysql -uroot -p radius
\end{lstlisting}

Pour spécifier un masque de sous-réseau, utiliser la notation CIDR.

Pour supprimer un NAS~:

\begin{lstlisting}
$ echo "DELETE FROM nas WHERE nasname='192.168.0.254' AND shortname='switch' AND secret='§\emph{mysecret}§'" | mysql -uroot -p radius
\end{lstlisting}

Pour ajouter un utilisateur~:

\begin{lstlisting}
$ echo "INSERT INTO radcheck(username,attribute,op,value) VALUES ('§\emph{userName}§','Cleartext-Password',':=','§\emph{userPassword}§')" | mysql -uroot -p radius
\end{lstlisting}

Penser à lui générer un certificat.

Pour supprimer un utilisateur~:

\begin{lstlisting}
$ echo "DELETE FROM radcheck WHERE username='§\emph{userName}§' AND attribute='Cleartext-Password' AND op=':=' AND value='§\emph{userPassword}§'" | mysql -uroot -p radius
\end{lstlisting}
