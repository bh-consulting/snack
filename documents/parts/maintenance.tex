\section{Maintenance}

\subsection{Ajouter un client NAS}

Pour ajouter un client'nas' (Switch ou autre équipement), qui pourra dès lors interroger le radius~:

\begin{alltt}
\$ echo "INSERT INTO nas VALUES (nasname, shortname, secret) VALUES \\('192.168.0.254','switch','poil');" | mysql -uroot -p radius
\end{alltt}

\textbf{Note:} Pour spécifier un netmask, utiliser la valeur 192.168.0.0/24

Pour supprimer un NAS~:
\begin{alltt}
\$ echo "DELETE FROM nas WHERE nasname='192.168.0.254' AND shortname='switch' \\AND secret='poil';" | mysql -uroot -p radius
\end{alltt}


Pour ajouter un utilisateur~:
\begin{alltt}
\$ echo "INSERT INTO radcheck(username,attribute,op,value) VALUES ('userName',\\'Cleartext-Password',':=','userPassword');" | mysql -uroot -p radius
\end{alltt}
Puis lui générer un certificat.

Pour supprimer un utilisateur~:
\begin{alltt}
\$ echo "DELETE FROM radcheck WHERE username='userName' AND attribute='Cleartext-Password'\\ AND op=':=' AND value='userPassword';" | mysql -uroot -p radius
\end{alltt}

