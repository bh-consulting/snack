\section{Certificats}
\subsection{Certificat racine et CRL}
\subsubsection{OpenSSL}

Pour l'instant, les utilisateurs ne peuvent s'authentifier qu'en utilisant un nom d'utilisateur et un mot de passe, avec du MD5. Pour pouvoir proposer des solutions qui apportent plus de sécurité, OpenSSL peut être utilisé pour générer des certificats.

Pour l'installer, en super-utilisateur sur le serveur~:

\begin{lstlisting}
$ sudo apt-get install openssl
\end{lstlisting}

Modifier le fichier de configuration \emph{/etc/ssl/openssl.cnf} pour changer le chemin par défaut des certificats créés (\emph{dossier\_certs}), ainsi que la valeur des clés \emph{certificate} et \emph{private\_key} (lignes 10 et 14)~:

\begin{lstlisting}
   [ CA_default ]

    dir             = dossier_certs              # Where everything is kept
    certs           = $dir/certs            # Where the issued certs are kept
    crl_dir         = $dir/crl              # Where the issued crl are kept
    database        = $dir/index.txt        # database index file.
    #unique_subject = no                    # Set to 'no' to allow creation of several ctificates with same subject.
    new_certs_dir   = $dir/newcerts         # default place for new certs.

    certificate     = $dir/users/ca_cert.pem       # The CA certificate
    serial          = $dir/serial           # The current serial number
    crlnumber       = $dir/crlnumber        # the current crl number must be commented out to leave a V1 CRL
    crl             = $dir/crl.pem          # The current CRL
    private_key     = $dir/private/ca_key.pem # The private key
\end{lstlisting}

Ajouter également ces deux sections à la fin du fichier, nécessaires pour supporter les clients Microsoft Windows~:

\begin{lstlisting}
[xpclient_ext]
extendedKeyUsage=1.3.6.1.5.5.7.3.2

[xpserver_ext]
extendedKeyUsage=1.3.6.1.5.5.7.3.1
\end{lstlisting}

\subsubsection{Création de l'environnement}

Créer les dossiers suivants~:

\begin{lstlisting}
$ mkdir -p dossier_certs/{certs,crl,private,users,newcerts}
\end{lstlisting}

Ainsi que les fichiers~:

\begin{lstlisting}
$ touch dossier_certs/index.txt
$ echo 01 > dossier_certs/crlnumber
\end{lstlisting}

\subsubsection{Génération du certificat racine}

Créer le certificat racine sans passphrase avec OpenSSL~:

\begin{lstlisting}
$ openssl genrsa -out dossier_certs/private/ca_key.pem 4096 
$ openssl req -new -key dossier_certs/private/ca_key.pem -subj /countryName=FR/stateOrProvinceName=France/localityName=Nancy/organizationName=BHConsulting/commonName=nom_entreprise/ -out dossier_certs/private/ca_req.pem
$ openssl ca -config /etc/ssl/openssl.cnf -create_serial -out dossier_certs/users/ca_cert.pem -days 3650 -batch -keyfile dossier_certs/private/ca_key.pem -selfsign -extensions v3_ca -infiles dossier_certs/private/ca_req.pem
\end{lstlisting}

Ignorer les éventuels messages \og{}\emph{unable to write 'random state'}\fg.

Générer la liste des CRL (\emph{Certificate Revocation List}) \textbf{dès à présent et après chaque création/révocation} de certificat. Une seule génération peut être faite pour plusieurs modifications. La procédure est expliquée dans la section suivante.

\subsubsection{Génération et mise à jour de la CRL}
\label{gen-crl}

Première génération ou mise à jour de la CRL~:

\begin{lstlisting}
$ openssl ca -config /etc/ssl/openssl.cnf -gencrl -out dossier_certs/crl/crl.pem
\end{lstlisting}

\subsubsection{Architecture nécessaire pour la CRL}

Mettre en place l'architecture nécessaire à la lecture de CRL par Radius (les chemins doivent être obligatoirement absolus)~:

\begin{lstlisting}
$ export HASH=$(openssl x509 -noout -hash -in dossier_certs/users/ca_cert.pem)
$ ln -s dossier_certs/users/ca_cert.pem dossier_certs/certs/$HASH.0
$ ln -s dossier_certs/crl/crl.pem dossier_certs/certs/$HASH.r0
\end{lstlisting}

\subsection{Certificats pour le Radius}
\subsubsection{Génération et signature}

Pour mettre en place les certificats de Radius, commencer par créer un fichier aléatoire~: 

\begin{lstlisting}
$ dd if=/dev/urandom of=dossier_certs/random count=2
$ openssl dhparam -check -text -5 1024 -out dossier_certs/dh
\end{lstlisting}

Générer et signer un certificat pour le Radius (penser à adapter la valeur du champ \emph{commonName} si le certificat n'est pas créé directement sur le serveur du Freeradius)~:

\begin{lstlisting}
$ openssl genrsa -out dossier_certs/private/radius_key.pem 4096
$ openssl req -config /etc/ssl/openssl.cnf -new -key dossier_certs/private/radius_key.pem -subj /countryName=FR/stateOrProvinceName=France/localityName=Nancy/organizationName=BHConsulting/commonName=$(hostname)/ -out dossier_certs/private/radius_req.pem -days 3650
$ openssl ca -config /etc/ssl/openssl.cnf -policy policy_anything -out dossier_certs/private/radius_cert.pem -batch -extensions xpserver_ext -infiles dossier_certs/private/radius_req.pem
\end{lstlisting}

\subsubsection{Configuration de Freeradius}

Adapter les paramètres de la section TLS du fichier \emph{/etc/freeradius/eap.conf} de la façon suivante (laisser la valeur par défaut pour les autres)~:

\begin{lstlisting}
tls {
	confdir = dossier_certs 
	certdir = ${confdir}
	cadir = ${confdir}/private
	usersdir = ${confdir}/users

	private_key_password =
	private_key_file = ${cadir}/radius_key.pem
	certificate_file = ${cadir}/radius_cert.pem
	CA_file = ${usersdir}/ca_cert.pem
	dh_file = ${certdir}/dh
	random_file = /dev/urandom
	fragment_size = 1024

	include_length = yes
	check_crl = yes
	CA_path = ${certdir}/certs
	check_cert_cn = %{User-Name}

	§\emph{[...]}§
}
\end{lstlisting}

Décommenter la ligne \emph{sql} et commenter la ligne \emph{files} de la section \emph{authorize} du fichier \emph{/etc/freeradius/sites-available/inner-tunnel}~:

\begin{lstlisting}
authorize {
    chap
    mschap
    suffix
    update control {
        Proxy-To-Realm := LOCAL
    }
    eap {
        ok = return
    }
    #files
    sql
    expiration
    logintime
    pap
}
\end{lstlisting}

\subsection{Certificats pour les clients}
\subsubsection{Génération et signature}

Génération d'une clé pour un utilisateur (\emph{utilisateur1})~:

\begin{lstlisting}
$ openssl genrsa -out dossier_certs/users/utilisateur1§§_key.pem 4096
\end{lstlisting}

Génération d'une requête pour faire signer le certificat, en prenant soin de définir le champ \emph{commonName} à l'aide de l'identifiant de l'utilisateur~:

\begin{lstlisting}
$ openssl req -config /etc/ssl/openssl.cnf -new -key dossier_certs/users/utilisateur1§§_key.pem -subj /countryName=FR/stateOrProvinceName=France/localityName=Nancy/organizationName=BHConsulting/commonName=utilisateur1/ -out dossier_certs/users/utilisateur1§§_req.pem -days 3650
\end{lstlisting}

Signature du certificat, avec le certificat racine créé précédemment~:

\begin{lstlisting}
$ openssl ca -config /etc/ssl/openssl.cnf -policy policy_anything -out dossier_certs/users/utilisateur1§§_cert.pem -batch -extensions xpclient_ext -infiles dossier_certs/users/utilisateur1§§_req.pem
\end{lstlisting}

Export au format P12 (qui contient à la fois la partie privée et publique du certificat client)~:

\begin{lstlisting}
$ openssl pkcs12 -password pass: -export -in dossier_certs/users/utilisateur1§§_cert.pem -inkey dossier_certs/users/utilisateurs1§§_key.pem -out dossier_certs/users/utilisateur1§§.p12 -clcerts
\end{lstlisting}

{\Large\Info} Le format P12 est requis par les clients Microsoft Windows, qui ne reconnaissent pas aussi facilement le format PEM. Puisque le format P12 fonctionne également sur les clients GNU/Linux et Mac OS X, il sera systématiquement utilisé dans la suite de cette installation. En outre, ils ont l'avantage de fusionner la partie privée et publique en un seul fichier, ce qui est pratique pour les certificats clients (mais inutilisable pour les certificats serveurs).

Penser à mettre à jour la CRL (cf. section~\ref{gen-crl} page~\pageref{gen-crl}).

Une erreur de ce type à la signature signifie que la valeur donnée pour le \emph{commonName} a déjà été utilisée par le passé (même si le certificat a été supprimé)~:

\begin{lstlisting}
failed to update database
TXT_DB error number 2
\end{lstlisting}

En cas d'erreur, il faut donc éditer le fichier \emph{dossier\_certs/index.txt} pour supprimer les reliquats des anciens certificats.

\subsubsection{Révocation}

Pour révoquer un certificat (irréversible)~: 

\begin{lstlisting}
$ openssl ca -config /etc/ssl/openssl.cnf -revoke dossier_certs/users/utilisateur1§§_cert.pem
\end{lstlisting}

Penser à mettre à jour la CRL (cf. section~\ref{gen-crl} page~\pageref{gen-crl}).

Pour vérifier qu'un certificat a bien été révoqué~:

\begin{lstlisting}
$ openssl verify -CApath dossier_certs/certs/ -crl_check dossier_certs/users/utilisateur1§§_cert.pem
\end{lstlisting}

{\Large\Info} Pour plus de manipulations avec OpenSSL, consulter la \emph{cheatsheet} du wiki de Samat.org\footnote{\url{http://wiki.samat.org/CheatSheet/OpenSSL}}.

\label{tests-certificats}
\subsubsection{Test avec un client}

Créer un répertoire pour les certificats (\emph{dossier\_certs\_utilisateur}) de l'utilisateur sur la machine cliente, pour y copier directement deux des fichiers qui ont été générés précédemment~:

\begin{enumerate}
\item \emph{dossier\_certs/users/ca\_cert.pem} (certificat publique de l'autorité de certification)
\item \emph{dossier\_certs/users/{\hlc{utilisateur1}}.p12} (certificats publique et privé de l'utilisateur)
%\item \emph{dossier\_certs/users/utilisateur1\_cert.pem} (certificat publique de l'utilisateur)
%\item \emph{dossier\_certs/users/utilisateur1\_key.pem} (certificat privé de l'utilisateur)
\end{enumerate}

La configuration du serveur actuelle permet à l'utilisateur de choisir entre un tunnel TLS, TTLS avec MD5 ou PEAP, selon ses exigences de sécurité. Les trois fichiers ci-dessous permettent de tester chacun des modes pour vérifier leur bon fonctionnement.

Pour TLS, il faut créer un fichier \emph{dot1x.certif.conf} contenant~:

\begin{lstlisting}
network={
    eap=TLS
    eapol_flags=0
    key_mgmt=IEEE8021X
    identity="utilisateur1"
    ca_cert="dossier_certs_utilisateur/ca_cert.pem"
    private_key="dossier_certs_utilisateur/utilisateur1§§.p12"
}
\end{lstlisting}

%À titre d'information, pour faire du TLS sous GNU/Linux sans passer par le format P12~:
%
%\begin{lstlisting}
%network={
%    eap=TLS
%    eapol_flags=0
%    key_mgmt=IEEE8021X
%    identity="utilisateur1"
%    ca_cert="dossier_certs_utilisateur/ca_cert.pem"
%    client_cert="dossier_certs_utilisateur/utilisateur1§§_cert.pem"
%    private_key="dossier_certs_utilisateur/utilisateur1§§_key.pem"
%}
%\end{lstlisting}

Pour TTLS avec un chiffrement MD5 dans le tunnel~:

\begin{lstlisting}
network={
    eap=TTLS
    eapol_flags=0
    key_mgmt=IEEE8021X
    identity="utilisateur1"
    password="pass_utilisateur1"
    ca_cert="dossier_certs_utilisateur/ca_cert.pem"
    phase2="auth=MD5"
}
\end{lstlisting}

Enfin, pour PEAP~:

\begin{lstlisting}
network={
    eap=PEAP
    eapol_flags=0
    key_mgmt=IEEE8021X
    identity="utilisateur1"
    password="pass_utilisateur1"
    ca_cert="dossier_certs_utilisateur/ca_cert.pem"
    phase2="auth=MSCHAPV2"
}
\end{lstlisting}

Côté serveur, démarrer Freeradius en mode verbeux comme expliqué dans la section~\ref{freeradius-x} page~\pageref{freeradius-x} (il faudra le relancer à chaque modification de sa configuration, en cas de problème).

Côté client, à l'aide de l'outil \emph{eapol\_test} (cf. section~\ref{eapol-test} page~\pageref{eapol-test}) et des fichiers de configuration indiqués ci-dessus, tester chacun des modes d'authentification~:

\begin{lstlisting}
$ ./eapol_test -c test.conf.eap -a§§192.168.1.10 -p1812 -s§§secret_radius -r0
\end{lstlisting}

La dernière ligne devrait, cette fois-ci, être \emph{SUCCESS} dans les trois cas.
