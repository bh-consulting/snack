\section{Radius}
L'implémentation de SNACK se base sur l'utilisation de Freeradius.
\subsection{Installation de FreeRadius}
\subsubsection{Paquets Debian/Ubuntu}

Installer les paquets correspondants, en root~:
\begin{verbatim}
# apt-get install freeradius freeradius-utils
\end{verbatim}

\subsubsection{Lien avec une base MySQL}

Installer MySQL et le paquet FreeRadius permettant de faire le lien entre le serveur d'authentification et la base de données~:
\begin{verbatim}
# apt-get install freeradius-mysql mysql-server mysql-client
\end{verbatim}

Penser à retenir le password root de la base mysql (nous prendrons pour la suite \emph{passrootsql}).

Créer les bases de données nécessaires pour FreeRadius à l'aide des schémas mis à disposition dans le paquet~:
\begin{alltt}
# echo "create database radius;" | mysql -u root -p
# echo "grant all on radius.* to radius@'\%' identified by '\emph{passrootsql}'; flush privileges;" | mysql -u root -p
# mysql -uroot -p radius < /etc/freeradius/sql/mysql/schema.sql
# mysql -uroot -p radius < /etc/freeradius/sql/mysql/nas.sql
\end{alltt}
Pour chacune de ces lignes, le mot de passe défini à l'étape précédente est demandé.

Création d'un utilisateur simple \emph{user1}, avec une authentification via un mot de passe non-chiffré (\emph{motdepasse})~:
\begin{alltt}
# echo "INSERT INTO radcheck(UserName,Attribute,op,Value) VALUES ('user1',\\'Cleartext-Password',':=','motdepasse');" | mysql -u root -p radius
\end{alltt}

Adapter le fichier \emph{/etc/freeradius/sql.conf}~:
\begin{alltt}
sql {
        database = "mysql"
        driver = "rlm_sql_\$\{database\}"
        server = "localhost"
        login = "root"
        password = "\emph{passrootsql}"
        radius_db = "radius"
        acct_table1 = "radacct"
        acct_table2 = "radacct"
        postauth_table = "radpostauth"
        authcheck_table = "radcheck"
        authreply_table = "radreply"
        groupcheck_table = "radgroupcheck"
        groupreply_table = "radgroupreply"
        usergroup_table = "radusergroup"
        deletestalesessions = yes
        sqltrace = no
        sqltracefile = \$\{logdir\}/sqltrace.sql
        num_sql_socks = 5
        connect_failure_retry_delay = 60
        readclients = yes
        nas_table = "nas"
        \$INCLUDE sql/\$\{database\}/dialup.conf
}
\end{alltt}

Dans le fichier \emph{/etc/freeradius/radiusd.conf}, décommenter les 2 lignes suivantes~:
\begin{verbatim}
$INCLUDE sql.conf
$INCLUDE sql/mysql/counter.conf
\end{verbatim}

Créer un virtualhost Radius~:
\begin{verbatim}
# cd /etc/freeradius/sites-available
# cp default radius.bh-consulting.net
# ln -s /etc/freeradius/sites-available/radius.bh-consulting.net\
  /etc/freeradius/sites-enabled/
\end{verbatim}

Modifier le virtualhost en éditant le fichier \emph{/etc/freeradius/sites-available/radius.bh-consulting.net}~:
\begin{verbatim}
authorize {
        preprocess
        chap
	mschap
        suffix
        sql
        expiration
        logintime
        pap
}

authenticate {
        Auth-Type PAP {
                pap
        }

        Auth-Type CHAP {
                chap
        }

	Auth-Type MS-CHAP {
		mschap
	}

        eap
}

preacct {
        preprocess
        acct_unique
        suffix
}

accounting {
        detail
        radutmp
        sql
}

session {
        radutmp
        sql
}

post-auth {
        sql
        # sql_log
        exec

        Post-Auth-Type REJECT {
                attr_filter.access_reject
        }
}

...
\end{verbatim}

\subsection{Test d'authentification}
Pour ce test, il nous faut ajouter le client dans la table des NAS admis. Soit l'IP du client : 192.168.1.2.
\begin{alltt}
# echo "insert into nas(nasname,shortname,secret) values ('\emph{192.168.1.2}','client',\\'poil'); " | mysql -uroot -p radius
\end{alltt}

Lancer le serveur Radius en mode debug sur la machine serveur (avec l'adresse IP \texttt{192.168.1.10})~:
\begin{verbatim}
# freeradius -X
\end{verbatim}
Si le serveur se plaint que le port est 'already in use', verifier qu'une instance de radius n'est pas déjà en train de tourner.

Côté client (avec \emph{freeradius-utils} d'installé et l'IP \texttt{192.168.1.2}), tenter une authentification avec l'utilisateur créé dans la base de données.\\
Pour rappel, l'IP du serveur radius: 192.168.1.10~:
\begin{verbatim}
# radtest user1 motdepasse 192.168.1.10 0 poil
\end{verbatim}

Si l'authentification est réussie, le Radius est prêt à être utilisé dans le cadre des deux sections suivantes.
\begin{comment}
\subsection{Contrôle d'accès au réseau}
\subsubsection{Protocole 802.1x}

Le contrôle d'accès au réseau permet de déterminer si un utilisateur a le droit de relier sa machine au réseau ou non. Concrétement, le port du commutateur correspondant sera bloqué par défaut, et ne commencera à laisser passer le trafic qu'en cas d'authentification réussie. Il déterminera le VLAN à utiliser pour le port avant de l'ouvrir, en fonction de l'utilisateur reconnu. Ce mode de fonctionnement est permis par le protocole 802.1x décrit dans la section \ref{dot1x} page \pageref{dot1x}, qui décrit les interactions avec le serveur Radius.

\subsubsection{Configurer le commutateur}

Enregistrer le serveur Radius (avec le secret comme clé)~:
\begin{verbatim}
Switch# conf t
Switch(config)# aaa new-model
Switch(config)# radius-server host 192.168.1.1 auth-port 1812 key poil
\end{verbatim}

Configurer un port en dot1x, en suivant la procédure décrite dans la section \ref{dot1x} page \pageref{dot1x}.

Ajouter une règle d'authentification via le Radius pour le dot1x~:
\begin{verbatim}
Switch(config)# aaa authentication dot1x poil group radius
\end{verbatim}

\subsubsection{Configurer le client}

Le supplicant permet au client de s'authentifier auprès du commutateur, pour que celui-ci puisse interroger le serveur Radius au moment de la connexion physique.

Installer un supplicant~:
\begin{verbatim}
# apt-get install wpa_supplicant
\end{verbatim}

Créer le fichier \emph{/etc/wpa\_supplicant/dot1x.conf}~:
\begin{verbatim}
ctrl_interface=/var/run/wpa_supplicant
ctrl_interface_group=0
eapol_version=1
ap_scan=0

network={
        key_mgmt=IEEE8021X
        eap=MD5
        identity="user1"
        password="motdepasse"
}
\end{verbatim}

\subsubsection{Test du 802.1x via Radius}

Après avoir lancé Radius sur le serveur, lancer le supplicant sur le client~:
\begin{verbatim}
# wpa_supplicant -c /etc/wpa_supplicant/dot1x.conf -D wired -i eth0
\end{verbatim}

Relier physiquement le client au commutateur au port configuré en dot1x, et vérifier que le port passe en autorisé.

\subsubsection{Assigner un numéro de VLAN à la volée}

TODO: Ajouts dans radreply

\subsection{Authentification sur les équipements actifs}
\subsubsection{Configuration du commutateur}

Enregistrer le serveur Radius~:
\begin{verbatim}
Switch# conf t
Switch(config)# aaa new-model
Switch(config)# radius-server host 192.168.1.1 auth-port 1812 key poil
\end{verbatim}

Ajouter une règle d'authentification via le Radius pour le login en telnet~:
\begin{verbatim}
Switch(config)# aaa authentication login poil group radius local
\end{verbatim}

Activer l'authentification en telnet~:
\begin{verbatim}
Switch(config)# line vty 0 15
Switch(config-line)# login authentication poil
\end{verbatim}

Créer un utilisateur local au cas où le serveur Radius ne soit pas accessible~:
\begin{verbatim}
Switch(config)# username user1 password pass1
\end{verbatim}

Ajouter une règle d'authentification via le Radius pour le login en console~:
\begin{verbatim}
Switch(config)# aaa authentication enable poil group radius enable
\end{verbatim}

Activer l'authentification en console~:
\begin{verbatim}
Switch(config)# line console 0
Switch(config-line)# login authentication poil
\end{verbatim}

Créer un mot de passe local au cas où le serveur Radius ne soit pas accessible~:
\begin{verbatim}
Switch(config)# enable password pass1
\end{verbatim}

\subsubsection{Configuration du serveur Radius}

Pour le mode enable, il faut autoriser des caractères spéciaux dans le fichier \textit{sql.conf} :
\begin{verbatim}
safe-characters = "@0123456789abcdefghijklmnopqrstuvwxyzABCDEFGHIJKLMNOPQRSTUVWXYZ0123456789.-_: /$"
\end{verbatim}

Pour le mode enable, il faut ajouter un utilisateur particulier dans la base de donnée :
\begin{verbatim}
insert into radcheck (username,attribute,value) values ('$enab15$','Cleartext-Password','pass1');
\end{verbatim}

\subsubsection{Test d'authentification sur le commutateur}

Accéder au commutateur en telnet depuis le client~:
\begin{verbatim}
$ telnet user1@192.168.1.20
\end{verbatim}

Saisir \emph{motdepasse} en guise de mot de passe.

(TODO: test en console)

\end{comment}
