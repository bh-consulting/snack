\section{Radius}
\subsection{Installation de Freeradius}
\subsubsection{Paquets Debian/Ubuntu}

Installer les paquets correspondants, en super-utilisateur~:

\begin{lstlisting}
# apt-get install freeradius freeradius-utils
\end{lstlisting}

\subsubsection{Lien avec une base MySQL}

Installer MySQL et le paquet Freeradius permettant de faire le lien entre le serveur d'authentification et la base de données~:

\begin{lstlisting}
# apt-get install freeradius-mysql mysql-server mysql-client
\end{lstlisting}

Penser à retenir le mot de passe super-utilisateur de la base MySQL (\emph{passrootsql}).

Créer les bases de données nécessaires pour Freeradius à l'aide des schémas mis à disposition dans le paquet~:

\begin{lstlisting}
# echo "create database radius" | mysql -u root -p
# echo "grant all on radius.* to radius@localhost identified by '§\emph{passradiussql}§'; flush privileges" | mysql -u root -p
# mysql -uroot -p radius < /etc/freeradius/sql/mysql/schema.sql
# mysql -uroot -p radius < /etc/freeradius/sql/mysql/nas.sql
\end{lstlisting}

Pour chacune de ces lignes, le mot de passe défini à l'étape précédente est demandé.

Création d'un utilisateur simple \emph{user1}, avec une authentification via un mot de passe non-chiffré (\emph{motdepasse})~:

\begin{lstlisting}
# echo "INSERT INTO radcheck(UserName,Attribute,op,Value) VALUES ('§\emph{user1}§','Cleartext-Password',':=','§\emph{motdepasse}§')" | mysql -u root -p radius
\end{lstlisting}

Adapter le fichier \emph{/etc/freeradius/sql.conf}~:
\begin{lstlisting}
sql {
        database = "mysql"
        driver = "rlm_sql_${database}"
        server = "localhost"
        login = "radius"
        password = "§\emph{passradiussql}§"
        radius_db = "radius"
        acct_table1 = "radacct"
        acct_table2 = "radacct"
        postauth_table = "radpostauth"
        authcheck_table = "radcheck"
        authreply_table = "radreply"
        groupcheck_table = "radgroupcheck"
        groupreply_table = "radgroupreply"
        usergroup_table = "radusergroup"
        deletestalesessions = yes
        sqltrace = no
        sqltracefile = ${logdir}/sqltrace.sql
        num_sql_socks = 5
        connect_failure_retry_delay = 60
        readclients = yes
        nas_table = "nas"
        $INCLUDE sql/${database}/dialup.conf
}
\end{lstlisting}

Dans le fichier \emph{/etc/freeradius/radiusd.conf}, décommenter les deux lignes suivantes~:

\begin{lstlisting}
$INCLUDE sql.conf
$INCLUDE sql/mysql/counter.conf
\end{lstlisting}

Créer un virtualhost Radius~:
\begin{lstlisting}
# cd /etc/freeradius/sites-available
# cp default radius.bh-consulting.net
# ln -s /etc/freeradius/sites-available/radius.bh-consulting.net /etc/freeradius/sites-enabled/
\end{lstlisting}

Modifier le virtualhost en éditant le fichier \emph{/etc/freeradius/sites-available/radius.bh-consulting.net}~:

\begin{lstlisting}
server radius.bh-consulting.net{

authorize {
        preprocess
        chap
        mschap
        suffix
	eap{
	    ok = return
	}
        sql
        expiration
        logintime
        pap
}

authenticate {
        Auth-Type PAP {
                pap
        }

        Auth-Type CHAP {
                chap
        }

	Auth-Type MS-CHAP {
		mschap
	}

        eap
}

preacct {
        preprocess
        acct_unique
        suffix
}

accounting {
        detail
        radutmp
        sql
}

session {
        radutmp
        sql
}

post-auth {
        sql
        # sql_log
        exec

        Post-Auth-Type REJECT {
                attr_filter.access_reject
        }
}

§\emph{[...]}§
}
\end{lstlisting}

Attention à ne pas oublier la première ligne \emph{server radius.bh-consulting.net\{}.

De la même façon, dans la section \emph{authorize} du fichier \emph{/etc/freeradius/sites-enabled/default}, commenter \emph{files} et décommenter \emph{sql} de cette façon~:

\begin{lstlisting}
authorize {
        preprocess
        chap
        mschap
        suffix
        eap{
            ok = return
        }
        # files
        sql
        expiration
        logintime
        pap
}

§\emph{[...]}§
\end{lstlisting}

\subsection{Test d'authentification}
\subsubsection{Test local}

Pour ce test, il nous faut ajouter le client (\emph{192.168.1.2}) dans la table des NAS admis~:

\begin{lstlisting}
# echo "INSERT INTO nas(nasname,shortname,secret) VALUES ('§\emph{192.168.1.2}§','client','§\emph{mysecret}§')" | mysql -uroot -p radius
\end{lstlisting}

Lancer le serveur Radius (\emph{192.168.1.10}) en mode debogue, et sans détachement, sur la machine serveur~:

\begin{lstlisting}
# freeradius -X
\end{lstlisting}

Si le serveur répond que le est \emph{already in use}, vérifier qu'une instance de Radius n'est pas déjà en train de tourner (par exemple à cause d'un service~: \texttt{service freeradius stop}).

Côté client, installer le paquet suivant~:

\begin{lstlisting}
# apt-get install freeradius-utils
\end{lstlisting}

Puis tenter une authentification avec l'utilisateur créé dans la base de données~:

\begin{lstlisting}
# radtest §\emph{user1}§ §\emph{motdepasse}§ §\emph{192.168.1.10}§ 0 §\emph{mysecret}§
\end{lstlisting}

Si l'authentification est réussie, le Radius est prêt à être utilisé dans le cadre des sections suivantes.

\subsubsection{Test distant}

Il est possible de simuler la présence d'un NAS pour tester le bon fonctionnement de Radius avec le programme \emph{eapol}, livré comme une partie de la suite \emph{wpa\_supplicant}. Le client utilisé pour ce test doit bien sûr être connu du serveur (cf. ci-dessus)~:

\begin{lstlisting}
$ eapol_test -c dot1x.conf.eap -a§\emph{192.168.1.10}§ -p1812 -s§\emph{mysecret}§ -r0
\end{lstlisting} 

Le fichier de configuration (\emph{dot1x.conf.eap}) est au format \emph{wpa\_supplicant}.

À titre d'exemple, pour EAP-MD5~:

\begin{lstlisting}
ctrl_interface=/var/run/wpa_supplicant
ctrl_interface_group=0
eapol_version=1
ap_scan=0

network={
        key_mgmt=IEEE8021X
        eap=MD5
        identity="§\emph{user1}§"
        password="§\emph{motdepasse}§"
}
\end{lstlisting}
