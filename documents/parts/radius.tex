\section{Radius}
L'implémentation de SNACK se base sur l'utilisation de Freeradius.
\subsection{Installation de FreeRadius}
\subsubsection{Paquets Debian/Ubuntu}

Installer les paquets correspondants, en root~:
\begin{verbatim}
# apt-get install freeradius freeradius-utils
\end{verbatim}

\subsubsection{Lien avec une base MySQL}

Installer MySQL et le paquet FreeRadius permettant de faire le lien entre le serveur d'authentification et la base de données~:
\begin{verbatim}
# apt-get install freeradius-mysql mysql-server mysql-client
\end{verbatim}

Penser à retenir le password root de la base mysql (nous prendrons pour la suite \emph{passrootsql}).

Créer les bases de données nécessaires pour FreeRadius à l'aide des schémas mis à disposition dans le paquet~:
\begin{alltt}
# echo "create database radius;" | mysql -u root -p
# echo "grant all on radius.* to radius@localhost identified by '\emph{passradiussql}'; \textbackslash \\flush privileges;" | mysql -u root -p
# mysql -uroot -p radius < /etc/freeradius/sql/mysql/schema.sql
# mysql -uroot -p radius < /etc/freeradius/sql/mysql/nas.sql
\end{alltt}
Pour chacune de ces lignes, le mot de passe défini à l'étape précédente est demandé.

Création d'un utilisateur simple \emph{user1}, avec une authentification via un mot de passe non-chiffré (\emph{motdepasse})~:
\begin{alltt}
# echo "INSERT INTO radcheck(UserName,Attribute,op,Value) VALUES ('user1',\\'Cleartext-Password',':=','motdepasse');" | mysql -u root -p radius
\end{alltt}

Adapter le fichier \emph{/etc/freeradius/sql.conf}~:
\begin{alltt}
sql {
        database = "mysql"
        driver = "rlm_sql_\$\{database\}"
        server = "localhost"
        login = "radius"
        password = "\emph{passradiussql}"
        radius_db = "radius"
        acct_table1 = "radacct"
        acct_table2 = "radacct"
        postauth_table = "radpostauth"
        authcheck_table = "radcheck"
        authreply_table = "radreply"
        groupcheck_table = "radgroupcheck"
        groupreply_table = "radgroupreply"
        usergroup_table = "radusergroup"
        deletestalesessions = yes
        sqltrace = no
        sqltracefile = \$\{logdir\}/sqltrace.sql
        num_sql_socks = 5
        connect_failure_retry_delay = 60
        readclients = yes
        nas_table = "nas"
        \$INCLUDE sql/\$\{database\}/dialup.conf
}
\end{alltt}

Dans le fichier \emph{/etc/freeradius/radiusd.conf}, décommenter les 2 lignes suivantes~:
\begin{verbatim}
$INCLUDE sql.conf
$INCLUDE sql/mysql/counter.conf
\end{verbatim}

Créer un virtualhost Radius~:
\begin{verbatim}
# cd /etc/freeradius/sites-available
# cp default radius.bh-consulting.net
# ln -s /etc/freeradius/sites-available/radius.bh-consulting.net\
  /etc/freeradius/sites-enabled/
\end{verbatim}

Modifier le virtualhost en éditant le fichier \emph{/etc/freeradius/sites-available/radius.bh-consulting.net}~:
\begin{verbatim}
server radius.bh-consulting.net{
authorize {
        preprocess
        chap
        mschap
        suffix
	eap{
	    ok = return
	}
        sql
        expiration
        logintime
        pap
}

authenticate {
        Auth-Type PAP {
                pap
        }

        Auth-Type CHAP {
                chap
        }

	Auth-Type MS-CHAP {
		mschap
	}

        eap
}

preacct {
        preprocess
        acct_unique
        suffix
}

accounting {
        detail
        radutmp
        sql
}

session {
        radutmp
        sql
}

post-auth {
        sql
        # sql_log
        exec

        Post-Auth-Type REJECT {
                attr_filter.access_reject
        }
}

...
}
\end{verbatim}
Attention a bien ajouter "server radius.bh-consulting.net\{", section qui englobe le reste du fichier.

De la même façon, dans la section 'authorize' du fichier \emph{/etc/freeradius/sites-enabled/default}, commenter 'files' et décommenter 'sql'.
\begin{verbatim}
authorize {
        preprocess
        chap
        mschap
        suffix
        eap{
            ok = return
        }
        # files
        sql
        expiration
        logintime
        pap
}
...
\end{verbatim}


\subsection{Test d'authentification}
\subsubsection{Test local}
Pour ce test, il nous faut ajouter le client dans la table des NAS admis. Soit l'IP du client : 192.168.1.2.
\begin{alltt}
# echo "insert into nas(nasname,shortname,secret) values ('\emph{192.168.1.2}','client',\\'mysecret'); " | mysql -uroot -p radius
\end{alltt}

Lancer le serveur Radius en mode debug sur la machine serveur (avec l'adresse IP \texttt{192.168.1.10})~:
\begin{verbatim}
# freeradius -X
\end{verbatim}
Si le serveur se plaint que le port est 'already in use', verifier qu'une instance de radius n'est pas déjà en train de tourner.

Côté client (avec \emph{freeradius-utils} d'installé et l'IP \texttt{192.168.1.2}), tenter une authentification avec l'utilisateur créé dans la base de données.\\
Pour rappel, l'IP du serveur radius: 192.168.1.10~:
\begin{verbatim}
# radtest user1 motdepasse 192.168.1.10 0 mysecret
\end{verbatim}

Si l'authentification est réussie, le Radius est prêt à être utilisé dans le cadre des sections suivantes.

\subsubsection{Test distant}
Il est possible de simuler la présence d'un NAS pour tester le bon fonctionnement du radius avec le programme 'eapol', livré comme une partie de la suite wpa\_supplicant. Le client utilisé pour ce test dois bien sur être connu du serveur (cf ci-dessus). L'appel est fait avec la commande : 
\begin{verbatim}
$ eapol_test -c dot1x.conf.eap -a192.168.1.10 -p1812 -smysecret -r0
\end{verbatim} 
Le fichier de configuration, ici \emph{dot1x.conf.eap} est au format wpa\_supplicant. A titre d'exemple, pour EAP-MD5~:
\begin{verbatim}
ctrl_interface=/var/run/wpa_supplicant
ctrl_interface_group=0
eapol_version=1
ap_scan=0

network={
        key_mgmt=IEEE8021X
        eap=MD5
        identity="user1"
        password="motdepasse"
}
\end{verbatim}




