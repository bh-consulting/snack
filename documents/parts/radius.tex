\section{Radius}
\subsection{Installation de Freeradius}
\subsubsection{Paquets Debian/Ubuntu}

Installer les paquets correspondants, en super-utilisateur~:

\begin{lstlisting}
$ sudo apt-get install freeradius freeradius-utils
\end{lstlisting}

\subsubsection{Lien avec une base MySQL}

Installer MySQL et le paquet Freeradius permettant de faire le lien entre le serveur d'authentification et la base de données~:

\begin{lstlisting}
$ sudo apt-get install freeradius-mysql mysql-server mysql-client
\end{lstlisting}

Penser à retenir le mot de passe super-utilisateur de la base MySQL (\emph{pass\_root\_sql}).

Créer les bases de données nécessaires pour Freeradius à l'aide des schémas mis à disposition dans le paquet (en n'oubliant pas de définir le mot de passe \emph{pass\_radius\_sql} correspondant à l'utilisateur MySQL qu'utilisera Freeradius)~:

\begin{lstlisting}
$ echo "CREATE DATABASE radius" | mysql -u root -p§§pass_root_sql
$ echo "GRANT ALL ON radius.* TO radius@localhost IDENTIFIED BY 'pass_radius_sql'; flush privileges" | mysql -u root -p§§pass_root_sql
$ sudo cat /etc/freeradius/sql/mysql/schema.sql | mysql -uroot -p§§pass_root_sql radius
$ sudo cat /etc/freeradius/sql/mysql/nas.sql | mysql -uroot -p§§pass_root_sql radius
\end{lstlisting}

{\Large\Info} Penser à systématiquement adapter les extraits de commandes ou de configuration proposés qui apparaissent {\hlc{en surbrillance}}.

Afin que Freeradius utilise les tables MySQL pour sa liste d'utilisateurs comme pour sa liste de NAS (\emph{Network Access Server}), adapter le fichier \emph{/etc/freeradius/sql.conf} (la ligne \emph{password} devrait être à changer, et la ligne \emph{readclients} à décommenter)~:
\begin{lstlisting}
sql {
        database = "mysql"
        driver = "rlm_sql_${database}"
        server = "localhost"
        login = "radius"
        password = "pass_radius_sql"
        radius_db = "radius"
        acct_table1 = "radacct"
        acct_table2 = "radacct"
        postauth_table = "radpostauth"
        authcheck_table = "radcheck"
        authreply_table = "radreply"
        groupcheck_table = "radgroupcheck"
        groupreply_table = "radgroupreply"
        usergroup_table = "radusergroup"
        deletestalesessions = yes
        sqltrace = no
        sqltracefile = ${logdir}/sqltrace.sql
        num_sql_socks = 5
        connect_failure_retry_delay = 60
        readclients = yes
        nas_table = "nas"
        $INCLUDE sql/${database}/dialup.conf
}
\end{lstlisting}

Pour le fichier SQL soit pris en compte, décommenter les deux lignes suivantes dans le fichier \emph{/etc/freeradius/radiusd.conf}~:

\begin{lstlisting}
$INCLUDE sql.conf
$INCLUDE sql/mysql/counter.conf
\end{lstlisting}

Créer un hôte Radius virtuel (\emph{snack} peut être remplacé par un autre nom) en copiant le contenu suivant dans un fichier \emph{/etc/freeradius/sites-available/snack}~:

\begin{lstlisting}
server snack {
	authorize {
		preprocess
		suffix

		eap {
			ok = return
		}
		sql

		expiration
		logintime
	}

	authenticate {
		eap
	}

	preacct {
		preprocess
		acct_unique
		suffix
	}

	accounting {
		detail
		radutmp
		sql
	}

	session {
		radutmp
		sql
	}

	post-auth {
		sql

		Post-Auth-Type REJECT {
			attr_filter.access_reject
		}
	}

	pre-proxy {
	}

	post-proxy {
		eap
	}
}
\end{lstlisting}

Activer l'hôte virtuel dans Freeradius~:

\begin{lstlisting}
$ sudo ln -s /etc/freeradius/sites-available/snack /etc/freeradius/sites-enabled/
\end{lstlisting}

L'autoriser à écouter par défaut en ajoutant cette ligne à la fin des \textbf{deux} sections \emph{listen} du fichier \emph{/etc/freeradius/radiusd.conf}~:

\begin{lstlisting}
    virtual_server = snack
\end{lstlisting}

Enfin, supprimer la configuration par défaut pour ne pas avoir de collisions~:

\begin{lstlisting}
$ sudo rm /etc/freeradius/sites-enabled/default
\end{lstlisting}

\subsection{Test d'authentification}
\subsubsection{Test local}

Ce test peut être fait entre deux machines, l'une faisant office de serveur d'authentification avec le Freeradius (\emph{192.168.1.10}), et l'autre faisant office de client/NAS (\emph{192.168.1.2}), avec un outil d'authentification.

\label{ajout-nas}
Ajouter le client (\emph{client1}) dans la table des NAS admis du serveur qui vient d'être configuré~:

\begin{lstlisting}
$ echo "INSERT INTO nas(nasname, shortname, secret) VALUES ('192.168.1.2', 'client1', 'secret_radius')" | mysql -uradius -p§§pass_radius_sql radius
\end{lstlisting}

\label{ajout-utilisateur-md5}
Créer un utilisateur simple \emph{utilisateur1}, qui pourra s'authentifier en 802.1x avec un mot de passe non-chiffré (\emph{pass\_utilisateur1})~:

\begin{lstlisting}
$ echo "INSERT INTO radcheck(UserName, Attribute, op, Value) VALUES ('utilisateur1', 'Cleartext-Password', ':=', 'pass_utilisateur1')" | mysql -u radius -p§§pass_radius_sql radius
\end{lstlisting}

\label{freeradius-x}
Lancer le serveur Radius en mode verbeux, et sans détachement~:

\begin{lstlisting}
$ sudo freeradius -X
\end{lstlisting}

Si le serveur s'arrête sur \og~\emph{Failed binding to authentication address * port 1812{}: Address already in use}~\fg{}, vérifier qu'une instance de Radius n'est pas déjà en train de tourner (par exemple à cause d'un service~: \texttt{service freeradius stop}). Il devrait bloquer le terminal sur \og~\emph{Ready to process requests.}\fg{}.

Côté client, installer le paquet suivant~:

\begin{lstlisting}
$ sudo apt-get install freeradius-utils
\end{lstlisting}

{\Large\Info} Un client de type GNU/Linux Debian (ou Ubuntu) sera utilisé pour valider chacune des étapes de cette documentation. Une fois que tout aura été validé, l'infrastructure sera prête à accueillir des clients GNU/Linux comme Microsoft Windows ou Mac OS X.

Puis tenter une authentification avec l'utilisateur créé dans la base de données~:

\begin{lstlisting}
$ radtest utilisateur1 pass_utilisateur1 192.168.1.10 0 secret_radius
\end{lstlisting}

Si l'authentification est réussie, un message \og\emph{rad\_recv: Access-Accept packet from host}\fg{} s'affiche côté client et un message \og\emph{Sending Access-Accept of id}\fg{} s'affiche côté serveur. Si un \emph{Access-Reject} est reçu côté client, consulter la trace affichée sur le serveur pour déterminer le problème.

\subsubsection{Test avec encapsulation EAP}

Le test précédent a permis de valider la configuration du serveur Radius, grâce à un simple échange de messages au format Radius. Ce second test permet d'ajouter une encapsulation de la requête dans un message EAP, qui sera contenu dans le message Radius. C'est le comportement qu'auront les commutateurs, pour relayer les demandes au serveur d'authentification.

\label{eapol-test}
Pour récupérer l'outil de test, il faut l'extraire des sources de \emph{wpa\_supplicant} et le compiler sur le client~:

\begin{lstlisting}
$ wget http://hostap.epitest.fi/releases/wpa_supplicant-2.0.tar.gz
$ tar xf wpa_supplicant-2.0.tar.gz
$ cd wpa_supplicant-2.0/wpa_supplicant/
$ cp defconfig .config
$ sed '/CONFIG_EAPOL_TEST/c CONFIG_EAPOL_TEST=y' -i .config
$ sudo apt-get install libnl-dev
$ make eapol_test
\end{lstlisting}

\label{auth-md5}
Créer un fichier d'authentification (\emph{dot1x.md5.conf}) en MD5, pour l'utilisateur \emph{utilisateur1}~:

\begin{lstlisting}
network={
        key_mgmt=IEEE8021X
        eap=MD5
        identity="utilisateur1"
        password="pass_utilisateur1"
}
\end{lstlisting}

Penser à lancer le Freeradius en mode verbeux (cf. section~\ref{freeradius-x} page~\pageref{freeradius-x})~:

Tester de nouveau la connexion~:

\begin{lstlisting}
$ ./eapol_test -c dot1x.md5.conf -a§§192.168.1.10 -p1812 -s§§secret_radius -r0
\end{lstlisting} 

L'affichage d'un message \og{}\emph{eapol\_sm\_cb: success=1}\fg{} dans les dernières lignes de la trace de \emph{eapol\_test} côté client confirme la réussite de l'authentification (il faut ignorer la toute dernière ligne \emph{FAILURE} qui prendra du sens par la suite). En cas d'échec, la valeur de \emph{success} est zéro. Côté serveur Radius, les messages sont les mêmes que lors du test précédent.

Si ces deux tests ont été concluants, la suite du tutoriel peut être suivie.
